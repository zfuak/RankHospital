\documentclass[12pt]{article}
\usepackage{fullpage,graphicx,psfrag,amsmath,amsfonts,verbatim}
\usepackage[small,bf]{caption}
\usepackage{amsthm}
\usepackage{hyperref}
\usepackage{bbm} % for the indicator function to look good
\usepackage{color}
\usepackage{mathtools}
\usepackage{fancyhdr} % for the header
\usepackage{booktabs} % for regression table display (toprule, midrule, bottomrule)
\usepackage{adjustbox} % for regression table display
\usepackage{threeparttable} % to use table notes
\usepackage{natbib} % for bibliography
\input newcommand.tex
\bibliographystyle{apalike}
% \setlength{\parindent}{0pt} % remove the automatic indentation

\title{L'Hopital's (Selection) Rule:\\{\large {An Empirical Bayes Application to French Hospital Efficiency}}}
\author{Fu Zixuan \\{\small {Supervised by Thierry Magnac}}}
\date{July 4, 2024}

\begin{document}
\maketitle
\thispagestyle{empty}
\begin{abstract}
    \noindent  Something interesting\\

    % \noindent\textbf{Keywords:} \\

    \bigskip
\end{abstract}

\newpage
\thispagestyle{empty}
\tableofcontents
\newpage

\setcounter{page}{1}
\section{Introduction}

It is almost of human nature to compare, rank and select. And competition, be
it good or bad, emerges in the wake. As invidious as ranking and selection can
be, in many cases it is one of the driving forces behind improvement in
performances. The society itself is constantly constructing league table as
well. It rewards the meritorious and question or even punishes the
unsatisfactory. The measure based on which rank is constructed ranges from
teacher's evaluation \citep{chetty2014measuring}, communities' mobility index
\citep{chetty2018impacts} to firm discrimination \citep{kline2022systemic}.

The present article extends the practice to the health sectors. To be more
specific, it studies the labor efficiency across all hospitals in France. By
exploring a comprehensive database called \textit{The Annual Statistics of
    Health Establishments (SAE)} of French hospitals, I first construct a measure
of labor efficiency. Then based on the estimates, we compare the public and
private hospitals by selecting the top-performing units. I borrow from the
recent developments in Empirical Bayes method to achieve the comparison.

I found that out of the top 20\% best performing hospitals, there are roughly 5
times more private units than the public, adjusted by the number of hospitals
in each category. The difference is more pronounced when I also control for the
expected number of wrongly selected. The takeaway is that public hospitals are
in general less efficient than private ones. While the conclusion is in line
with that of \cite{croiset2024hospitals} that, now we have a granular
perspective on the performance comparison.

The article bridges two fields of interests. The first one is on productivity
analysis. The most popular methods in the field are Data Envelopment Analysis
\citep{charnes1978measuring} and Stochastic Frontier Analysis
\citep{aigner1977formulation,meeusen1977efficiency}. Yet I abstract from both
of them and use the \textit{conditional input demand function} specification
stated in \cite{croiset2024hospitals}.\footnote{I refer the reader to
    \citet{croiset2024hospitals} for detailed reasons of adopting such an
    approach.} To put it simply, we estimate a linear function of how much labor
input is needed to produce a give list of 8 hospital outputs. I only focus on
the employment level of nurses because unlike medical doctors, this is a
category that do not suffer from a shortage of labor supply.

The second area of interests is the Empirical Bayes Methods. I lean on a series
of work by Jiaying Gu and Roger Koenker, chiefly the following two papers.
\citet{gu2017empirical} discussed the usefulness of estimating a prior
distribution in baseball batting average prediction. And
\citet{gu2023invidious} has formally defined the selection problem as a
compound decision on which the estimated prior can be of help as well.
\citet{kiefer1956consistency} has shown that non parametric maximun likelihood
estimation of the prior is feasible and consistent. The computation of NPMLE is
greatly improved by \citet{koenker2014convex} by leveraging the recent
development in convex optimization \citep{andersen2010mosek}. I will be using
the \verb+REBayes+ package \citep{koenker2017rebayes} in the estimation, which
is based on software \verb+MOSEK+ developed by \citet{andersen2010mosek}.

In \cite{croiset2024hospitals}, the authors argue that public hospital is less
efficient than private counterpart in the sense that it would need a smaller
size of personnel if it were to use the input demand function of the private
hospital, which is the main result of their counterfactuals.

Having roughly replicated the results after doubling the length of the panel,
the paper differentiates itself by utilizing classical panel data methods in
input demand function estimation, specifically the standard fixed-effect
estimation and GMM. Though it is straightforward to include individual fixed
effect in specification, estimation is not without challenge. For example, as
\citet{croiset2024hospitals} has correctly pointed out, within hospital
variation is much smaller than between group variation. The former may be
insufficient to obtain credible estimates. I extended the panel length in an
attempt to mitigate the problem. Secondly, the strict exogeneity assumption
required by the standard within group estimation is questionable. A natural way
to relax it is to use the first difference GMM estimator proposed by
\citet{arellano1991some}. The high persistency in the regressors poses another
challenge of weak instruments. In response,
\citet{arellano1995another,blundell1998initial}'s system GMM is modified and
implemented, and the estimation results are used for the rest of the section.

The benefit of the panel data estimator is that it gives us an estimate of the
underlying heterogeneity, which opens door to individual comparisons. However,
the fixed effect estimates are generally noisy, rendering the ensuing decision
maker hand wavy in making choices. The EB methods are proposed in an attempt to
rectify the situation by empirically estimating the prior distribution of the
fixed effect.

For example, in \cite{gu2023invidious}, we are given the task of selecting the
top 20\% fixed effect denoted by $\theta_i$. If the $\theta_i$ follows a
distribution $G$, this is to say we are selecting those $\theta_i>G^{-1}(0.8)$.
The decision rule for individual $i$ is an indicator function $\delta_i$,
determining whether $i$ belongs to selection set. The task naturally falls
under the compound decision framework pioneered by \cite{herbert1956empirical}
if we define the loss function of the selection problem in such a way that
takes into account the results of all the individual decisions $\delta_i$.
\begin{equation*}
    \delta^* = \argmin_{\delta} \E_G\E_{\theta|\hat{\theta}}\pa{L_n}.
\end{equation*}
Since we don't know the true value $\theta$, we minimize the expected compound loss  $L_n$ over the distribution of $\theta$ given the observed $\hat{\theta}$.

In addition to the capacity constraint of the top 20\%, \citet{gu2023invidious}
further controls for the number of Type II mistakes made in the selection
process. The false discovery rate (FDR) constraint is imposed to ensure that
the expected number of wrongly selected units is below a certain level. The FDR
constraint is a measure of the proportion of false positives among all the
selected units defined as $\p(h_i=0|\delta_i=1)\le \gamma$.

Being interested in the top performing French hospitals, I define my selection
problem as \textit{Left tail selection} because the goal is to choose the
bottom 20\% of the hospital fixed effect $\theta_i$. A smaller $\theta_i$
indicates that less labor input is needed to produce the same amount of output,
as compared to hospitals with higher $\theta_i$.

It is worth mentioning that classical empirical Bayes method assumes a
parametric form of the prior distribution $G$ which is computationally more
attractive. Yet thanks to fast convex optimization algorithms, the
non-parametric maximum likelihood estimation is now both feasible and
efficient. Nevertheless, we are completely free from imposing any parametric
assumption. In fact, there are two \textit{layers} of distribution. The lower
hierarchy is the prior $G$ with $\theta\sim G$ while the higher hierarchy is
$\hat{\theta}|\theta \sim P_{\theta}$. It is when $P_{\theta}$ belongs to the
exponential family that the \citet{lindsay1995mixture} results hold. Usually in
application, we need to impose assumptions or perform some transformation such
that $P_{\theta}$ is normal. This kind of procedure is often questionable.
Often times, researchers resort to asymptotics to justify the normality
assumption, which may not be valid in small samples.

The rest of the paper is organized as follows. Section 2 briefly describes the
data and lays out the reduced form estimation of the input demand function,
treating the number of nurses as the dependent variable and a list of 9 output
measures as the regressors. It then applies the classical panel data estimators
to the same specification, distinguishing between whether strict exogeneity is
assumed. In section 3, I introduce the compound decision framework and the
method to non parametrically estimate $G$. In section 4, I specifically define
the selection problem following the framework of \cite{gu2023invidious}.
Section 5 follows with a comparison of the different selection outcome. I try
to draw preliminary conclusion on the comparative performance of public and
private hospitals. Section 6 discusses potential issues and concludes.

\section{Data and Estimation}
\subsection{Data}
The data we used is called \textit{The Annual Statistics of Health
    Establishments
    (SAE)}\footnote{\href{https://data.drees.solidarites-sante.gouv.fr/explore/dataset/708_bases-statistiques-sae/information/}{La
        Statistique annuelle des établissements (SAE)}}. It is a comprehensive,
mandatory administrative survey and the primary source of data on all health
establishments in France. We primarily exploited the report of healthcare
output (a list of 10 output measure) and labor input (registered and assistant
nurses). The panel covers 9 years from 2013 to 2022, with 2020 missing due to
the pandemic. The SAE data only distinguishes 3 types of units based on legal
status. \textit{
    \begin{enumerate}
        \item Public hospitals
        \item Private for-profit hospitals
        \item Private non-profit hospitals
    \end{enumerate}}
Following \cite{croiset2024hospitals}, I further single out/distinguish the \textit{public teaching hospitals} from the public hospitals since it is intrinsically different from others in the French healthcare system.

As shown in Table \ref{tab:hospital_count}, The number of hospitals in normal
public, private for-profit, private non-profit are roughly equal and stable
over the years. With respect to the teaching hospitals, it is worth mentioning
that they not only provide treatments like other types of hospitals but spend a
significant amount of resources on doctor training and research as well. Since
teaching hospitals have more missions on top of the regular healthcare
provision, it is natural that they are in general larger in size. This latter
point can be seen much more clearly we present the hospital's output share.
Despite being relatively few in number, their share of output is quite
substantial. The difference is more pronounced after being adjusted by the
number of hospitals as shown in Table \ref{tab:output}.

Moreover, we see that each type of hospital differs in terms of the mix of
services they provide. For example, emergency care is mostly taken care of by
public hospitals and private hospitals are strong in medical sessions.

\begin{table}
    \centering
    % latex table generated in R 4.2.1 by xtable 1.8-4 package
% Tue Jun 18 19:02:10 2024
\begin{tabular}{rrrrrrr}
  \toprule
 & AN & Teaching & Normal Public & Private For Profit & Private Non Profit & Total \\ 
  \midrule
1 & 2013.00 & 198 & 1312 & 1305 & 1382 & 4197.00 \\ 
  2 & 2014.00 & 201 & 1274 & 1293 & 1349 & 4117.00 \\ 
  3 & 2015.00 & 211 & 1275 & 1297 & 1349 & 4132.00 \\ 
  4 & 2016.00 & 212 & 1266 & 1297 & 1313 & 4088.00 \\ 
  5 & 2017.00 & 211 & 1249 & 1297 & 1306 & 4063.00 \\ 
  6 & 2018.00 & 214 & 1247 & 1296 & 1288 & 4045.00 \\ 
  7 & 2019.00 & 214 & 1236 & 1287 & 1281 & 4018.00 \\ 
  8 & 2021.00 & 219 & 1222 & 1293 & 1264 & 3998.00 \\ 
  9 & 2022.00 & 220 & 1220 & 1296 & 1259 & 3995.00 \\ 
   \bottomrule
\end{tabular}

    \caption{Number of hospitals in each category, 2013-2022}
    \label{tab:hospital_count}
\end{table}

\begin{table}\fontsize{10pt}{12pt}\selectfont
    \centering
    \begin{threeparttable}[b]

        % latex table generated in R 4.2.1 by xtable 1.8-4 package
% Sun Jun 23 16:16:45 2024
\begin{tabular}{llllll}
  \toprule
Output & Teaching & Normal Public & Private For Profit & Private Non Profit & Total \\ 
  \midrule
STAC inpatient & 25.17\% & 43.09\% & 23.64\% & 8.1\% & 100\% \\ 
  STAC oupatient & 18.4\% & 19.46\% & 52.95\% & 9.18\% & 100\% \\ 
  Sessions & 14.49\% & 21.96\% & 34.4\% & 29.16\% & 100\% \\ 
  Outpatient Consultations & 36.8\% & 52.45\% & 0.23\% & 10.52\% & 100\% \\ 
  Emergency & 21.4\% & 60.06\% & 13.37\% & 5.17\% & 100\% \\ 
  Follow-up care and Long-term care & 7.6\% & 19.47\% & 37.95\% & 34.98\% & 100\% \\ 
  Home hospitalization & 13\% & 17.38\% & 12.4\% & 57.22\% & 100\% \\ 
  Psychiatry stays & 6.53\% & 62.26\% & 12.93\% & 18.28\% & 100\% \\ 
   \bottomrule
\end{tabular}

        \caption{Hospital share of output, 2013-2022}
        \label{tab:nonadjusted}
    \end{threeparttable}
\end{table}

\begin{table}\fontsize{10pt}{12pt}\selectfont
    \centering
    \begin{threeparttable}[b]
        % latex table generated in R 4.2.1 by xtable 1.8-4 package
% Sun Jun 23 15:56:55 2024
\begin{tabular}{llllll}
  \toprule
Output & Teaching & Normal Public & Private For Profit & Private Non Profit & Total \\ 
  \midrule
STAC inpatient & 66.98\% & 19.29\% & 10.25\% & 3.48\% & 100\% \\ 
  STAC oupatient & 57.91\% & 10.29\% & 27.13\% & 4.67\% & 100\% \\ 
  Sessions & 50.12\% & 12.7\% & 20.18\% & 16.99\% & 100\% \\ 
  Outpatient Consultations & 77.69\% & 18.64\% & 0.08\% & 3.59\% & 100\% \\ 
  Emergency & 62.02\% & 29.26\% & 6.31\% & 2.41\% & 100\% \\ 
  Follow-up care and Long-term care & 33.5\% & 14.37\% & 27.31\% & 24.82\% & 100\% \\ 
  Home hospitalization & 47.83\% & 10.75\% & 7.46\% & 33.96\% & 100\% \\ 
  Psychiatry stays & 29.65\% & 47.38\% & 9.6\% & 13.37\% & 100\% \\ 
   \bottomrule
\end{tabular}

        \caption{Hospital share of output weighted by the number of hospitals, 2013-2022}
        \label{tab:output}
        \begin{tablenotes}[para,flushleft]
            \footnotesize
            For example, the value $a_{ij}$ where $i$ is STAC inpatient and $j$ is teaching hospitals, is calculated by $a_{ij}= \frac{\text{Number of STAC inpatient  in teaching hospitals}}{\text{Share of teaching hospitals}\times \text{Total number of STAC inpatient}}$.
        \end{tablenotes}

    \end{threeparttable}
\end{table}

\subsection{Estimation}

\paragraph{Regression without individual fixed effect}

Let $\log(x_{it})$ be the number of nurses in hospital $i$ at time $t$, and
$\log(y_{it})$ denote a vector of output levels. I estimate
\begin{equation}
    \log(x_{it}) = \beta_0 + \beta_1 \log(y_{it}) + \varepsilon_{it}
\end{equation}

First, having performed the regression separately for each type of hospital, it
is without surprise that teaching hospitals have very different coefficients,
as shown in Table \ref{reg_sep}. In addition to the differences in descriptive
statistics from the last section, this intrinsic difference in input demand
functions or equivalently in production function is another sign that teaching
hospitals may not be directly comparable to other types of hospitals. For this
reason, I will exclude teaching hospitals from the subsequent analysis.

\begin{table}
    \centering
    
\begingroup
\centering
\begin{tabular}{lcccc}
   \tabularnewline \midrule \midrule
   Dependent Variable: & \multicolumn{4}{c}{log(ETP\_INF)}\\
                      & Teaching    & Public      & Forprofit   & Nonprofit \\   
   Model:             & (1)         & (2)         & (3)         & (4)\\  
   \midrule
   \emph{Variables}\\
   Constant           & 3.28$^{a}$  & 1.38$^{a}$  & 1.40$^{a}$  & 1.00$^{a}$\\   
                      & (0.328)     & (0.262)     & (0.095)     & (0.149)\\   
   log(SEJHC\_MCO)    & 0.108$^{b}$ & 0.331$^{a}$ & 0.261$^{a}$ & 0.344$^{a}$\\   
                      & (0.042)     & (0.048)     & (0.015)     & (0.034)\\   
   log(SEJHP\_MCO)    & 0.132$^{a}$ & 0.078$^{a}$ & 0.048$^{a}$ & 0.046$^{c}$\\   
                      & (0.032)     & (0.013)     & (0.011)     & (0.027)\\   
   log(SEANCES\_MED)  & 0.060$^{a}$ & 0.051$^{a}$ & 0.075$^{a}$ & 0.094$^{a}$\\   
                      & (0.020)     & (0.007)     & (0.006)     & (0.016)\\   
   log(CONSULT\_EXT)  & 0.017       & 0.025$^{a}$ & -0.003      & 0.001\\   
                      & (0.014)     & (0.008)     & (0.011)     & (0.012)\\   
   log(PASSU)         & 0.049$^{a}$ & -0.009      & 0.033$^{a}$ & 0.025$^{b}$\\   
                      & (0.011)     & (0.008)     & (0.005)     & (0.010)\\   
   log(ENTSSR)        & 0.058$^{a}$ & 0.052$^{a}$ & 0.057$^{a}$ & 0.118$^{a}$\\   
                      & (0.013)     & (0.008)     & (0.008)     & (0.019)\\   
   log(SEJ\_HAD)      & 0.022       & 0.028$^{a}$ & 0.049$^{a}$ & -0.011\\   
                      & (0.027)     & (0.007)     & (0.018)     & (0.022)\\   
   log(SEJ\_PSY)      & 0.026$^{b}$ & 0.070$^{a}$ & 0.084$^{a}$ & 0.045\\   
                      & (0.011)     & (0.010)     & (0.018)     & (0.046)\\   
   \midrule
   \emph{Fit statistics}\\
   Observations       & 1,123       & 5,260       & 4,415       & 2,604\\  
   R$^2$              & 0.779       & 0.860       & 0.742       & 0.754\\  
   \midrule \midrule
   \multicolumn{5}{l}{\emph{Clustered (FI) standard-errors in parentheses}}\\
   \multicolumn{5}{l}{\emph{Signif. Codes: a: 0.01, b: 0.05, c: 0.1}}\\
\end{tabular}
\par\endgroup



    \caption{Separate estimation of input demand function, lagged value as IV, 2013-2022}
    \label{reg_sep}
\end{table}

By excluding the teaching hospitals from estimation, it becomes more reasonable
to assume that all hospitals share the same set of coefficients, giving rise to
the pooled regression results shown in Table \ref{reg_dummy_iv_ex}.
\begin{table}
    \centering
    
\begingroup
\centering
\begin{tabular}{lcc}
   \tabularnewline \midrule \midrule
   Dependent Variable: & \multicolumn{2}{c}{Nurses}\\
                           & Dummy          & Dummy IV \\   
   Model:                  & (1)            & (2)\\  
   \midrule
   \emph{Variables}\\
   Constant                & 1.51$^{***}$   & 1.50$^{***}$\\   
                           & (0.025)        & (0.028)\\   
   STAC inpatient          & 0.291$^{***}$  & 0.290$^{***}$\\   
                           & (0.004)        & (0.005)\\   
   STAC outpatient         & 0.048$^{***}$  & 0.048$^{***}$\\   
                           & (0.003)        & (0.004)\\   
   Medical sessions        & 0.068$^{***}$  & 0.068$^{***}$\\   
                           & (0.002)        & (0.002)\\   
   External consultations  & 0.025$^{***}$  & 0.028$^{***}$\\   
                           & (0.002)        & (0.002)\\   
   Emergency               & 0.019$^{***}$  & 0.018$^{***}$\\   
                           & (0.001)        & (0.001)\\   
   Long-term \& follow-up  & 0.066$^{***}$  & 0.067$^{***}$\\   
                           & (0.002)        & (0.002)\\   
   Home care               & 0.026$^{***}$  & 0.025$^{***}$\\   
                           & (0.002)        & (0.003)\\   
   Psychiatric care        & 0.072$^{***}$  & 0.071$^{***}$\\   
                           & (0.003)        & (0.004)\\   
   Private Forprofit       & -0.258$^{***}$ & -0.245$^{***}$\\   
                           & (0.024)        & (0.027)\\   
   Private Nonprofit       & -0.178$^{***}$ & -0.160$^{***}$\\   
                           & (0.020)        & (0.022)\\   
   \midrule
   \emph{Fit statistics}\\
   Observations            & 14,067         & 12,279\\  
   R$^2$                   & 0.820          & 0.821\\  
   \midrule \midrule
   \multicolumn{3}{l}{\emph{Heteroskedasticity-robust standard-errors in parentheses}}\\
   \multicolumn{3}{l}{\emph{Signif. Codes: ***: 0.01, **: 0.05, *: 0.1}}\\
\end{tabular}
\par\endgroup



    \label{reg_dummy_iv_ex}
\end{table}

\paragraph{Regression with individual fixed effect}

Let $\log(x_{it})$ and $\log(y_{it})$ the same as before. In addition, let
$\theta_i$ be the fixed effect of hospital $i$. One interpretation of
$\theta_i$ is the measure of labor \emph{inefficiency}. The smaller the
$\theta_i$, the more efficient the hospital is in labor use. The estimate of
$\theta_i$ will be used to rank and select the hospitals in the next section.
The specification now is
\begin{equation}
    \log(x_{it}) = \beta_0 + \beta_1 \log(y_{it}) + \theta_i+ \varepsilon_{it}
\end{equation}

I considered 5 types of estimator, within-group, first difference, fist
difference GMM, system GMM and just identified system GMM. For the sake of
exposition, the linear specification takes the general form \[
    y_{it} = x_{it} \beta +\theta_i + \epsilon_{it}\quad \text{where} \quad E[\epsilon_{it}|x_{i1},\ldots, x_{it-1},\theta_i]=0.
\]
The system GMM makes use of two types of moment conditions. The first one is
that from the first difference GMM estimator,
\[E[x_{i,t-2}(\Delta y_{it}-\beta\Delta x_{it})]\]
where lagged $x_{i,t-2}$ serves as instrument for $\Delta x_{it}$. If the
persistency in $x_{it}$ is high, that is to say $x_{it}=\alpha
    x_{i,t-1}+\eta_{it}$ with $\alpha$ close to 1. Then the reduced form
relationship between $\Delta x_{it}$ and $x_{i,t-2}$ is
\[\Delta x_{it} = (\alpha-1)\alpha x_{i,t-2}+\alpha \eta_{i,t-1}+\eta_{i,t}\]
posing the problem of weak instrument.

The second moment condition makes another assumption, requiring that the
correlation between $x_{it}$ and $\theta_i$ is the same as that between
$x_{i,t-1}$ and $\theta_i$,
\begin{equation*}
    \E[\Delta x_{i,t-1}(y_{it}-\beta x_{it})] \quad \text{if}\quad \E\bra{\Delta x_{i,t-1}(\theta_i+\varepsilon_{i,t})}=0
\end{equation*} where the current level $x_{it}$ is instrumented by lagged first difference $\Delta x_{i,t-1}$.

It is obvious that there's a large difference between the first two estimators
and the GMM ones, a sign that the exogeneity assumption may not be valid.
Second, the first difference GMM estimates gives mull results, possibly due to
weak instruments. Though the estimate from system GMM looks more hopeful, the
sargan-hansen test almost rejects over-identification null hypothesis for sure,
indicating that some moment conditions are not in accordance with each other.
The fifth just identified GMM only makes use of the second type of moment
conditions from system GMM, abstracting from over-identification issue. Though
the issues of weak instrument, rejection of over-identification are intriguing
problems, I will set them aside for future investigation since the focus of the
paper is more empirical bayes application. Believing in the validity of the
assumption $\E\bra{\Delta x_{i,t-1}(\theta_i+\varepsilon_{i,t})}=0$, I will
take as given the estimation results from the last column of Table \ref{} and
proceed to the next section.

\begin{table}
    \label{tab:reg_wg_fd_gmm}
    
\begingroup
\centering
\begin{tabular}{lccc}
   \tabularnewline \midrule \midrule
   Dependent Variable:                 & \multicolumn{3}{c}{Nurses}                                   \\
                                       & Within Group               & First Difference & System GMM   \\
   Model:                              & (1)                        & (2)              & (3)          \\
   \midrule
   \emph{Variables}                                                                                   \\
   STAC inpatient                      & $0.10^{***}$               & $0.07^{***}$     & $0.51^{***}$ \\
                                       & $(0.00)$                   & $(0.01)$         & $(0.02)$     \\
   STAC outpatient                     & $0.02^{***}$               & $0.01^{***}$     & $0.06^{***}$ \\
                                       & $(0.00)$                   & $(0.00)$         & $(0.02)$     \\
   Medical sessions                    & $0.02^{***}$               & $0.02^{***}$     & $0.04^{***}$ \\
                                       & $(0.00)$                   & $(0.00)$         & $(0.01)$     \\
   External consultations              & $0.00$                     & $0.00$           & $0.07^{***}$ \\
                                       & $(0.00)$                   & $(0.00)$         & $(0.01)$     \\
   Emergency                           & $0.01^{***}$               & $0.01$           & $-0.07^{**}$ \\
                                       & $(0.00)$                   & $(0.00)$         & $(0.03)$     \\
   Long-term              \& follow-up & $0.01^{***}$               & $0.01^{***}$     & $0.02$       \\
                                       & $(0.00)$                   & $(0.00)$         & $(0.02)$     \\
   Home care                           & $0.01^{***}$               & $0.02^{**}$      & $0.01$       \\
                                       & $(0.00)$                   & $(0.01)$         & $(0.02)$     \\
   Psychiatric care                    & $0.02^{***}$               & $0.01$           & $0.04$       \\
                                       & $(0.00)$                   & $(0.01)$         & $(0.03)$     \\
   \midrule

   \midrule
   \emph{Fit statistics}                                                                              \\
   n                                   & $1690$                     & $1690$           & $1690$       \\
   T                                   & $9$                        & $9$              & $9$          \\
   \midrule \midrule
   \multicolumn{4}{l}{\emph{Signif. Codes: ***: 0.01, **: 0.05, *: 0.1}}                              \\
\end{tabular}
\par\endgroup


\end{table}

\begin{table}
    \label{tab:reg_wg_fd_iv}
    
\begingroup
\centering
\begin{tabular}{lccc}
   \tabularnewline \midrule \midrule
   Dependent Variable:                 & \multicolumn{3}{c}{Nurses}                                   \\
                                       & Within Group               & First Difference & System GMM   \\
   Model:                              & (1)                        & (2)              & (3)          \\
   \midrule
   \emph{Variables}                                                                                   \\
   STAC inpatient                      & $0.10^{***}$               & $0.07^{***}$     & $0.74^{***}$ \\
                                       & $(0.00)$                   & $(0.01)$         & $(0.08)$     \\
   STAC outpatient                     & $0.02^{***}$               & $0.01^{***}$     & $-0.07$      \\
                                       & $(0.00)$                   & $(0.00)$         & $(0.04)$     \\
   Medical sessions                    & $0.02^{***}$               & $0.02^{***}$     & $0.07^{***}$ \\
                                       & $(0.00)$                   & $(0.00)$         & $(0.02)$     \\
   External consultations              & $0.00$                     & $0.00$           & $0.03$       \\
                                       & $(0.00)$                   & $(0.00)$         & $(0.02)$     \\
   Emergency                           & $0.01^{***}$               & $0.01$           & $-0.11^{*}$  \\
                                       & $(0.00)$                   & $(0.00)$         & $(0.05)$     \\
   Long-term              \& follow-up & $0.01^{***}$               & $0.01^{***}$     & $-0.04$      \\
                                       & $(0.00)$                   & $(0.00)$         & $(0.05)$     \\
   Home care                           & $0.01^{***}$               & $0.02^{**}$      & $0.04$       \\
                                       & $(0.00)$                   & $(0.01)$         & $(0.06)$     \\
   Psychiatric care                    & $0.02^{***}$               & $0.01$           & $-0.09$      \\
                                       & $(0.00)$                   & $(0.01)$         & $(0.19)$     \\
   \midrule

   \midrule
   \emph{Fit statistics}                                                                              \\
   n                                   & $1690$                     & $1690$           & $1690$       \\
   T                                   & $9$                        & $9$              & $9$          \\
   \midrule \midrule
   \multicolumn{4}{l}{\emph{Signif. Codes: ***: 0.01, **: 0.05, *: 0.1}}                              \\
\end{tabular}
\par\endgroup


\end{table}

\section{Empirical Bayes and Selection Problem}

\subsection{Compound decision framework}

The idea of compound decision is pioneered by \citet{robbins1956empirical},
which takes into account the consequences of all individual decisions. Consider
the case where each individual unit has an unobserved parameter $\theta_i$. We
are given a list of estimates $\hat{\theta}_i$ for each $\theta_i$.
\begin{equation*}
    \boldsymbol{\hat{\theta}}  =  (\hat{\theta}_1,\ldots, \hat{\theta}_n)\quad
    \text{where} \quad        \hat{\theta}_i | \theta_i \sim P_{\theta_i}
\end{equation*}
For the moment, I will be agnostic to the specific decision to make and denote the decision rule by $\delta$.
\begin{equation*}
    \delta(\boldsymbol{\hat{\theta}}) = (\delta_1(\boldsymbol{\hat{\theta}}), \ldots, \delta_n(\boldsymbol{\hat{\theta}}))
\end{equation*}

The next step is to define the loss function as the objective function to
minimize. Since I care about the \textbf{collective performance} of my
decision, I will define the loss function such that the attention to the
compound decision is reflected. A natural choice would be to aggregate the
individual losses. Therefore, the compound loss function is defined as
\begin{equation*}
    L_n(\theta, \delta(\boldsymbol{\hat{\theta}})) = \sum_{i=1}^n L(\theta_i, \delta_i(\hat{\theta})).
\end{equation*}
Correspondingly, the compound risk is defined as the expectation of compound loss
\begin{equation*}
    R_n(\theta, \delta(\boldsymbol{\hat{\theta}})) = \E_{\theta|\hat{\theta}}[L_n(\theta, \delta(\boldsymbol{\hat{\theta}}))]
\end{equation*}
We further restrict our attention to the separable decision rule $\delta(\boldsymbol{\hat{\theta}})=\{t(\hat{\theta}_1), \ldots, t(\hat{\theta}_n)\}$. In order to make the connection with the Bayesian view under which we assume
that $\theta\sim G$, we can rewrite the compound risk as
\begin{equation*}
    R_n(\theta, \delta(\boldsymbol{\hat{\theta}})) = \int \int L(\theta_i, t(\hat{\theta}_i))dP_{\theta_i}(\hat{\theta}_i)dG_n(\theta)
\end{equation*}
where $G_n(\theta)$ is the empirical distribution of $\theta$. \footnote{$E_{G_n}(f(x)) = 1/n \sum_i f(x_i)$}

The Frequentist and Bayesian views differ slightly here in the definition of
risk. The original compound decision formulation keeps the empirical
distribution $G_n$ in compound risk while the Bayesian risk replaces it with
the prior distribution $G$. On a side note, the two views are somewhat related
to the two assumptions in the fixed/random effect terminology, in the sense
that the fixed effect view treats $\theta_i$ as fixed unknown parameters while
the random effect view treats $\boldsymbol{\theta}$ as a random draw from a
distribution $G$. However, in our context, it has nothing to do with whether
$\theta_i$ is correlated with $x_{it}$.

The last step is to find the decision rule $\delta^*$ that minimizes the risk
\begin{equation}
    \delta^*=\argmin_\delta R_n(\theta, \delta(\boldsymbol{\hat{\theta}}))
\end{equation}
subject to any constraints that we may have. Since $G$ is unknown, whether we use $G_n$ or $G$ in risk does not matter much. In the rest of the section, I adopt the Bayesian risk as the objective of minimization and impose constraints relevant to the selection problem. Now I will turn to the non-parametric estimation of the prior distribution $G$.

\subsection{Estimate $G$}

\paragraph{Parametric $G$}
Most literature has imposed a parametric form of $G$. In the case of a Gaussian
$G$, recall the hierarchical model
\begin{align*}
    \hat{\theta}_i|\theta_i, \sigma_i \sim P_{\theta_i} \\
    \theta\sim \caln(\mu_\theta,\sigma^2_\theta)
\end{align*}
There are two hyperparameters to be estimated $\mu_\theta$ and $\sigma_\theta$.
% A common estimator would be 
% \begin{align*}
%     \hat{\mu}_\theta&=\frac{1}{N} \sum_{i}\hat{\theta}_i\\
%     \hat{\sigma}_\theta^2&=\frac{1}{N} \sum_{i} \bra{(\hat{\beta}_i-\hat{\mu}_\theta)^2-\sigma_i^2}
% \end{align*}

If we compare the performance of posterior mean estimator
$\theta^*=\E\bra{\theta|\hat{\theta}}$ with the original estimate
$\hat{\theta}$, \citet{james1992estimation} has shown that there's always an
improvement in the average performance if we assume $G$ is Gaussian and replace
it with an estimate $\hat{G}$. If we relax the normality assumption on $G$ and
adopt a NPMLE estimation as established by \citet{kiefer1956consistency}, there
could be further improvements. For example, \citet{jiang2009general} has proven
that a plugged in $\theta^*$ with a NPMLE $\hat{G}$ is asymptotically optimal
among all separable estimators. A comparison between the parametric and non
parametric $\hat{G}$ is demonstrated in \citet{gilraine2020new} on their
teacher value added application.

\paragraph{NPMLE $G$}

The initial NPMLE estimator defined in \citet{kiefer1956consistency} takes the
following form
\begin{equation*}
    \hat{G}=\argmin_{G\in \mathcal{G}} \set{-\sum_{i=1}^{n}\log g(y_i)|g(y_i)=\int  \p(y_i |\theta)dG(\theta) }
\end{equation*}
where $\p(y_i |\theta)$ is the probability density function of $y_i$ conditional on the true parameter $\theta$ $\longrightarrow$ $g(y_i)$ is the marginal pdf of $y_i$.

Though this is a convex optimization problem with strictly convex objective and
a convex constraint set, it is of infinite dimension. In order to solve the
primal problem, it is necessary to discretize. The algorithm proposed by
\citet{koenker2014convex} has taken advantage of the fixed point iteration
method in convex optimization \citep{andersen2010mosek}, thus greatly improved
the computation efficiency over the fixed point EM iteration method by
\citet{jiang2009general}.

\subsection{The selection problem}
The definition of the selection problem is taken from the work of
\citet{gu2023invidious}. Instead of focusing on the right tail of the
distribution, the top performers in my context corresponds to the left tail.
The task at hand is to select the bottom 20\% of the $\theta_i$ and compare the
share of public and private in the meritorious group. This is another
perspective/exercises on the public and private sectors different from that of
\citet{croiset2024hospitals}.

On top of the constraint on the size of the selected group (20\%), I further
impose a constraint on the number of false positive mistakes made in the
selection process. This leads to the false discovery constraint at level
$\gamma$,
\begin{equation*}
    \frac{\E_G\bra{h_i=0,\delta_i=1}}{\E_G\bra{\delta_i}} \le \gamma
\end{equation*} where $h_i=1\set{\theta_i<\theta_\alpha}$ is the indicator function of whether the unit $i$ is truly below the threshold $\theta_\alpha$. And $\delta_i=1$ when unit $i$ is selected.

All in all, we can formally define the selection problem as
\begin{equation*}
    L(\delta,\theta)=\sum h_i(1-\delta_i) +\tau_1\pa{\sum (1-h_i)\delta_i -\gamma \delta_i} + \tau_2 \pa{\sum \delta_i -\alpha n}
\end{equation*}
and the optimal decision rule is given by
\begin{equation*}
    \begin{split}
        \delta^*& =\argmin_\delta \E_G\E_{\theta|\hat{\theta}}\bra{L(\delta,\theta)}                                                                                                       \\
        &= \E_G \ \sum \E_{\theta|\hat{\theta}}(h_i)(1-\delta_i) +\tau_1\pa{\sum (1-\E_{\theta|\hat{\theta}}(h_i))\delta_i -\gamma \delta_i} + \tau_2 \pa{\sum \delta_i -\alpha n}            \\
        & =\E_G{\sum v_\alpha(\hat{\theta})(1-\delta_i) +\tau_1\pa{\sum (1-v_\alpha(\hat{\theta}))\delta_i -\gamma \delta_i} + \tau_2 \pa{\sum \delta_i -\alpha n}}
    \end{split}
\end{equation*}
Here, the term $\E_{\theta|\hat{\theta}}(h_i)$ is called \textbf{posterior tail probability}. It is the probability of $i$ being truly in the bottom $\alpha\%$ given the estimated $\hat{\theta}$.
This is a posterior statistics different from the posterior mean $\E_{\theta|\hat{\theta}}(\theta_i)$ because the variable inside the expectation $h_i=1\set{\theta_i<G^{-1}(\alpha)}$ is specific to the capacity constraint at $\alpha$ level.

% Our task at hand is to select the top $\alpha\%$ (e.g. 20\%) of the hospitals
% in terms of labor use efficiency. If $\theta_i$ represents a measure of
% \emph{inefficiency} which is the fixed effect term in the linear input demand
% function specified in Section 2. We want to select the top $\theta_i$s that is
% below than the $\alpha$ quantile of the population $\theta_i<G_n^{-1}(\alpha)$.
% Moreover, we want to subject the selection to another constraint which is the
% False Discovery Rate constraint at level $\gamma$, that is
% \begin{align*}
%     \p\bra{\theta_i>\theta_{\alpha}|\delta_i=1} & =\frac{\p\bra{\theta_i>\theta_{\alpha},\delta_i=1}}{\p\bra{\delta_i=1}}           \\
%                                                 & = \frac{\E_G\bra{1\set{\theta_i>\theta_{\alpha},\delta_i=1}}}{\E_G\bra{\delta_i}} \\
%                                                 & \le \gamma                                                                        \\
% \end{align*}
% Now we are in the position to write down the selection problem subject to the
% capacity constraint at $\alpha$ and FDR constraint at $\gamma$ level, with multiplier $\tau_1$ and $\tau_2$ respectively. We denote
% $\delta_i=1$ when unit $i$ is selected and $h_i=1\{\theta_i<\theta_\alpha\}=1$ when unit $i$ is truly below
% the threshold $\theta_\alpha =G^{-1}(\alpha)$. The compound loss
% function is defined as
% \begin{equation*}
%     L(\delta,\theta)=\sum h_i(1-\delta_i) +\tau_1\pa{\sum (1-h_i)\delta_i -\gamma \delta_i} + \tau_2 \pa{\sum \delta_i -\alpha n}
% \end{equation*}
% To minimize the compound risk is thus
% \begin{align*}
%     \min_{\delta} & \E_G\E_{\theta|\hat{\theta}}\bra{L(\delta,\theta)}                                                                                                        \\
%                   & =\E_G{\sum \E(h_i)(1-\delta_i) +\tau_1\pa{\sum (1-\E(h_i))\delta_i -\gamma \delta_i} + \tau_2 \pa{\sum \delta_i -\alpha n}}                               \\
%                   & =\E_G{\sum v_\alpha(\hat{\theta})(1-\delta_i) +\tau_1\pa{\sum (1-v_\alpha(\hat{\theta}))\delta_i -\gamma \delta_i} + \tau_2 \pa{\sum \delta_i -\alpha n}} \\
% \end{align*}
% where $v_\alpha(\hat{\theta})=\p(\theta<\theta_\alpha|\hat{\theta})$, which we called \textbf{posterior tail probability}. This is in contrast to the posterior mean widely used to shrink the estimate $\hat{\theta}_i$.
% For the moment, it may not immediately obvious how important the prior distribution $G$ is. I will further illustrate it in the next section.
% \subsection{Posterior tail probability}
% For each $\theta_i$, I observe a sequence of $Y_{it}$ coming from a
% longitudinal model
% \begin{equation*}
%     Y_{it} = \theta_i + \varepsilon_{it} \quad \varepsilon_{it} \sim \caln(0,\sigma_i^2) \quad (\theta_i,\sigma_i^2) \sim G
% \end{equation*}
% Neither $\theta_i$ nor $\sigma_i^2$ is known to us. But there are two sufficient statistics for $(\theta_i,\sigma_i^2)$.
% \begin{align*}
%     Y_i=\frac{1}{T_i}\sum_{t=1}^{T_i}Y_{it}           & \quad \text{where}\quad Y_i|\theta_i,\sigma_i^2 \sim \caln(\theta_i,\sigma_i^2/T_i)     \\
%     S_i=\frac{1}{T_i-1}\sum_{t=1}^{T_i}(Y_{it}-Y_i)^2 & \quad \text{where} \quad S_i|\sigma_i^2 \sim \Gamma(r_i= (T_i-1)/2,2\sigma_i^2/(T_i-1)) \\
% \end{align*}

% The tail probability $v_\alpha(y_i,s_i)$ given the two sufficient statistics is
% defined as
% \begin{align*}
%     v_\alpha(Y_i,S_i) & = P( \theta_i > \theta_{\alpha} | Y_i,S_i)                                                                                 \\
%                       & = \frac{{\int_{-\infty}^{\theta_{\alpha}} \Gamma(s_i|r_i,\sigma_i^2) f(y_i|\theta_i, \sigma_i^2) dG(\theta_i,\sigma_i^2)}}
%     {{\int_{-\infty}^{\infty} \Gamma(s_i|r_i,\sigma_i^2) f(y_i|\theta_i, \sigma_i^2) dG(\theta_i,\sigma_i^2)}}
% \end{align*}

% \paragraph{Write out the capacity constraint}
% \paragraph{Write out the FDR constraint}

% If our specification and assumptions on exogeneity are correct, the consistency
% of $\hat{\beta}$ is guaranteed by $N$'s asymptotic. However, our estimate of
% the fixed effect is
% \begin{align*}
%     \hat{\theta}_i & =\frac{1}{T}\sum(\theta_i+\varepsilon_{it}+x_{it}(\beta-\hat{\beta}))              \\
%                    & \overset{N\to \infty}{\longrightarrow} \theta_i+\frac{1}{T}\sum_t \varepsilon_{it} \\
% \end{align*}
% When $T$ is relatively small (or even fixed), I am not in a good position to use central limit theorem to claim that $\hat{\theta}_i \overset{d}{\to} \caln(\theta_i,\frac{\sigma_i^2}{T})$. A bold assumption that $\varepsilon_{it} \sim \caln(0,\sigma_i^2)$ will save me from the $T$ issue, which I will impose for the rest of the section (and abstract from whether that  for each $i$ is a testable/reasonable/feasible assumption).

% \section{Selection results}

% Although \cite{gu2023invidious} has presented the decision rule when
% $(\theta_i,\sigma_i^2)$ are unknown. The application assumes that $\sigma_i^2$
% is known/observable. In this section, I will compare the selection results
% under known variance $\sigma_i^2$ and estimated variance $S_i$.

% \paragraph{Known variance, and 4 rules}
% \paragraph{Unknown variance, and 4 rules}

% \section{Conclusion}

\newpage
\bibliography{ref.bib}

\appendix
\section{Data}
\subsection{Data selection}

The panel is first filtered by the following criteria
\begin{enumerate}
    \item the number of nurses is positive,
    \item at least one of STAC inpatient, STAC outpatient, Sessions is positive,
    \item the number of observations is larger than 6
\end{enumerate}
Second, I add one to every variable to avoid null value when taking log.

\section{Empirical Bayes}
\subsection{NPMLE $G$}
\citet{koenker2014convex} defined the primal problem as
\begin{equation*}
    \min_{f=dG}\set{-\sum_i \log g(y_i)\bigg |g(y_i) = T(f),\ K(f)=1,\ \forall i }
\end{equation*}
where $ T(f)=\int p(y_i |\theta)fd\theta $ and  $K(f)= \int f d\theta$.\\
By discretizing the support,
\begin{equation*}
    \min_{f=dG}\left\{-\sum_i \log g(y_i)\bigg |g=Af,\ {1^T}f=1\right\}
\end{equation*}
where $A_{ij}= p(y_i|\theta_j) $ and $ f = (f(\theta_1),f(\theta_2),\ldots,f(\theta_m))$.\\
It is straightforward to derive the dual problem
\begin{equation*}
    \max_{\lambda,\mu} \left\{ \sum_i \log \lambda_1(i) \bigg| A^T\lambda_1 < \lambda_2 1,\ (\lambda_1>0) \right\}
\end{equation*}
\end{document}
