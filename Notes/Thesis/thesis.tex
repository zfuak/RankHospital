\documentclass[12pt]{article}
\usepackage{fullpage,graphicx,psfrag,amsmath,amsfonts,verbatim}
\usepackage[small,bf]{caption}
\usepackage{amsthm}
\usepackage{hyperref}
\usepackage{bbm} % for the indicator function to look good
\usepackage{color}
\usepackage{mathtools}
\usepackage{fancyhdr} % for the header
\usepackage{booktabs} % for regression table display (toprule, midrule, bottomrule)
\usepackage{adjustbox} % for regression table display
\usepackage{threeparttable} % to use table notes
\usepackage{natbib} % for bibliography
\input newcommand.tex
\bibliographystyle{apalike}
% \setlength{\parindent}{0pt} % remove the automatic indentation

\title{L'Hopital's (Selection) Rule:\\ A Bayesian Application to French Hospitals}
\author{Fu Zixuan \\
    Supervised by Thierry Magnac \thanks{Last compiled on \today}}
\date{July 4, 2024}
\begin{document}
\maketitle

\begin{abstract}
    \noindent  Something interesting\\

    % \noindent\textbf{Keywords:} \\

    \bigskip
\end{abstract}

\newpage
\tableofcontents
\newpage

\section{Introduction}

It is almost of human nature to compare, rank and select. And competition, be
it good or bad, emerges in the wake. As invidious as ranking and selection can
be, in many cases it is one of the driving forces behind improvement in
performances. The society itself is constantly constructing league table as
well. It rewards the meritorious and question or even punishes the
unsatisfactory. The measure based on which rank is constructed ranges from
teacher's evaluation \citep{chetty2014measuring}, communities' mobility index
\citep{chetty2018impacts} to firm discrimination \citep{kline2022systemic}.

The present article extends the practice to the health sectors. To be more
specific, it studies the labor efficiency across all hospitals in France. By
exploring a comprehensive database called \textit{The Annual Statistics of
    Health Establishments (SAE)} of French hospitals, I first construct a measure
of labor efficiency. Then based on the estimates, we compare the public and
private hospitals by selecting the top-performing units. I borrow from the
recent developments in Empirical Bayes method to achieve the comparison.

I found that out of the top 20\% best performing hospitals, there are roughly 5
times more private units than the public, adjusted by the number of hospitals
in each category. The difference is more pronounced when I also control for the
expected number of wrongly selected. The takeaway is that public hospitals are
in general less efficient than private ones. While the conclusion is in line
with that of \cite{croiset2024hospitals} that, now we have a granular
perspective on the performance comparison.

The article bridges two fields of interests. The first one is on productivity
analysis. The most popular methods in the field are Data Envelopment Analysis
\citep{charnes1978measuring} and Stochastic Frontier Analysis
\citep{aigner1977formulation,meeusen1977efficiency}. Yet I abstract from both
of them and use the \textit{conditional input demand function} specification
stated in \cite{croiset2024hospitals}.\footnote{I refer the reader to
    \citet{croiset2024hospitals} for detailed reasons of adopting such an
    approach.} To put it simply, we estimate a linear function of how much labor
input is needed to produce a give list of 8 hospital outputs. I only focus on
the employment level of nurses because unlike medical doctors, this is a
category that do not suffer from a shortage of labor supply.

The second area of interests is the Empirical Bayes Methods. I lean on a series
of work by Jiaying Gu and Roger Koenker, chiefly the following two papers.
\citet{gu2017empirical} discussed the usefulness of estimating a prior
distribution in baseball batting average prediction. And
\citet{gu2023invidious} has formally defined the selection problem as a
compound decision on which the estimated prior can be of help as well.
\citet{kiefer1956consistency} has shown that non parametric maximun likelihood
estimation of the prior is feasible and consistent. The computation of NPMLE is
greatly improved by \citet{koenker2014convex} by leveraging the recent
development in convex optimization \citep{andersen2010mosek}. I will be using
the \verb+REBayes+ package \citep{koenker2017rebayes} in the estimation, which
is based on software \verb+MOSEK+ developed by \citet{andersen2010mosek}.

In \cite{croiset2024hospitals}, the authors argue that public hospital is less
efficient than private counterpart in the sense that it would need a smaller
size of personnel if it were to use the input demand function of the private
hospital, which is the main result of their counterfactuals.

Having roughly replicated the results after doubling the length of the panel,
the paper differentiates itself by including/adding the standard/classical
panel data methods in input demand function estimation, specifically the
standard fixed-effect estimation and GMM.

The benefit of the panel data estimator is that it gives us an estimate of the
underlying heterogeneity, which opens door to individual comparisons. However,
the fixed effect estimates are generally noisy, rendering the ensuing decision
maker hand wavy in making choices. The EB methods are proposed in an attempt to
rectify the situation by empirically estimating the prior distribution of the
fixed effect.

For example, in \cite{gu2023invidious}, we are given the task of selecting the
top 20\% fixed effect denoted by $\theta_i$. If the $\theta_i$ follows a
distribution $G$, this is to say we are selecting those $\theta_i>G^{-1}(0.8)$.
The decision rule for individual $i$ is an indicator function $\delta_i$,
determining whether $i$ belongs to selection set. The task naturally falls
under the compound decision framework pioneered by \cite{herbert1956empirical}
if we define the loss function of the selection problem in such a way that
takes into account the results of all the individual decisions $\delta_i$.
\begin{equation*}
    \delta^* = \argmin_{\delta} \E_G\E_{\theta|\hat{\theta}}\pa{L_n}.
\end{equation*}
Since we don't know the true value $\theta$, we minimize the expected compound loss  $L_n$ over the distribution of $\theta$ given the observed $\hat{\theta}$.

In addition to the capacity constraint of the top 20\%, \citet{gu2023invidious}
further controls for the number of Type II mistakes made in the selection
process. The false discovery rate (FDR) constraint is imposed to ensure that
the expected number of wrongly selected units is below a certain level. The FDR
constraint is a measure of the proportion of false positives among all the
selected units defined as $\p(h_i=0|\delta_i=1)\le \gamma$.

Being interested in the top performing French hospitals, I define my selection
problem as \textit{Left tail selection} because the goal is to choose the
bottom 20\% of the hospital fixed effect $\theta_i$. A smaller $\theta_i$
indicates that less labor input is needed to produce the same amount of output,
as compared to hospitals with higher $\theta_i$.

It is worth mentioning that classical empirical Bayes method assumes a
parametric form of the prior distribution $G$ which is computationally more
attractive. Yet thanks to fast convex optimization algorithms, the
non-parametric maximum likelihood estimation is now both feasible and
efficient. Nevertheless, we are completely free from imposing any parametric
assumption. In fact, there are two \textit{layers} of distribution. The lower
hierarchy is the prior $G$ with $\theta\sim G$ while the higher hierarchy is
$\hat{\theta}|\theta \sim P_{\theta}$. It is when $P_{\theta}$ belongs to the
exponential family that the \citet{lindsay1995mixture} results hold. Usually in
application, we need to impose assumptions or perform some transformation such
that $P_{\theta}$ is normal. This kind of procedure is often questionable.
Often times, researchers resort to asymptotics to justify the normality
assumption, which may not be valid in small samples.

The rest of the paper is organized as follows. Section 2 briefly describes the
data and lays out the reduced form estimation of the input demand function,
treating the number of nurses as the dependent variable and a list of 9 output
measures as the regressors. It then applies the classical panel data estimators
to the same specification, distinguishing between whether strict exogeneity is
assumed. In section 3, I introduce the compound decision framework and the
method to non parametrically estimate $G$. In section 4, I specifically define
the selection problem following the framework of \cite{gu2023invidious}.
Section 5 follows with a comparison of the different selection outcome. I try
to draw preliminary conclusion on the comparative performance of public and
private hospitals. Section 6 discusses potential issues and concludes.

\section{Data and Estimation}
\subsection{Data}
The data we used is called \textit{The Annual Statistics of Health
    Establishments
    (SAE)}\footnote{\href{https://data.drees.solidarites-sante.gouv.fr/explore/dataset/708_bases-statistiques-sae/information/}{La
        Statistique annuelle des établissements (SAE)}}. It is a comprehensive,
mandatory administrative survey and the primary source of data on all health
establishments in France. We primarily exploited the report of healthcare
output (a list of 10 output measure) and labor input (registered and assistant
nurses). The panel covers 9 years from 2013 to 2022, with 2020 missing due to
the pandemic. The SAE data only distinguishes 3 types of units based on legal
status. \textit{
    \begin{enumerate}
        \item Public hospitals
        \item Private for-profit hospitals
        \item Private non-profit hospitals
    \end{enumerate}}
Following \cite{croiset2024hospitals}, I further single out/distinguish the \textit{public teaching hospitals} from the public hospitals since it is intrinsically different from others in the French healthcare system.

As shown in Table \ref{tab:hospital_count}, The number of hospitals in normal
public, private for-profit, private non-profit are roughly equal and stable
over the years. With respect to the teaching hospitals, it is worth mentioning
that they not only provide treatments like other types of hospitals but spend a
significant amount of resources on doctor training and research as well. Since
teaching hospitals have more missions on top of the regular healthcare
provision, it is natural that they are in general larger in size. This latter
point can be seen much more clearly we present the hospital's output share.
Despite being relatively few in number, their share of output is quite
substantial. The difference is more pronounced after being adjusted by the
number of hospitals as shown in Table \ref{tab:output}.

Moreover, we see that each type of hospital differs in terms of the mix of
services they provide. For example, emergency care is mostly taken care of by
public hospitals and private hospitals are strong in medical sessions.

\begin{table}\label{tab:hospital_count}
    \centering
    % latex table generated in R 4.2.1 by xtable 1.8-4 package
% Tue Jun 18 19:02:10 2024
\begin{tabular}{rrrrrrr}
  \toprule
 & AN & Teaching & Normal Public & Private For Profit & Private Non Profit & Total \\ 
  \midrule
1 & 2013.00 & 198 & 1312 & 1305 & 1382 & 4197.00 \\ 
  2 & 2014.00 & 201 & 1274 & 1293 & 1349 & 4117.00 \\ 
  3 & 2015.00 & 211 & 1275 & 1297 & 1349 & 4132.00 \\ 
  4 & 2016.00 & 212 & 1266 & 1297 & 1313 & 4088.00 \\ 
  5 & 2017.00 & 211 & 1249 & 1297 & 1306 & 4063.00 \\ 
  6 & 2018.00 & 214 & 1247 & 1296 & 1288 & 4045.00 \\ 
  7 & 2019.00 & 214 & 1236 & 1287 & 1281 & 4018.00 \\ 
  8 & 2021.00 & 219 & 1222 & 1293 & 1264 & 3998.00 \\ 
  9 & 2022.00 & 220 & 1220 & 1296 & 1259 & 3995.00 \\ 
   \bottomrule
\end{tabular}

    \caption{Number of hospitals in each category, 2013-2022}
\end{table}

\begin{table}\label{tab:nonadjusted}\fontsize{10pt}{12pt}\selectfont
    \begin{threeparttable}[b]
        \centering
        % latex table generated in R 4.2.1 by xtable 1.8-4 package
% Sun Jun 23 16:16:45 2024
\begin{tabular}{llllll}
  \toprule
Output & Teaching & Normal Public & Private For Profit & Private Non Profit & Total \\ 
  \midrule
STAC inpatient & 25.17\% & 43.09\% & 23.64\% & 8.1\% & 100\% \\ 
  STAC oupatient & 18.4\% & 19.46\% & 52.95\% & 9.18\% & 100\% \\ 
  Sessions & 14.49\% & 21.96\% & 34.4\% & 29.16\% & 100\% \\ 
  Outpatient Consultations & 36.8\% & 52.45\% & 0.23\% & 10.52\% & 100\% \\ 
  Emergency & 21.4\% & 60.06\% & 13.37\% & 5.17\% & 100\% \\ 
  Follow-up care and Long-term care & 7.6\% & 19.47\% & 37.95\% & 34.98\% & 100\% \\ 
  Home hospitalization & 13\% & 17.38\% & 12.4\% & 57.22\% & 100\% \\ 
  Psychiatry stays & 6.53\% & 62.26\% & 12.93\% & 18.28\% & 100\% \\ 
   \bottomrule
\end{tabular}

        \caption{Hospital share of output, 2013-2022}
    \end{threeparttable}
\end{table}

\begin{table}\label{tab:output}\fontsize{10pt}{12pt}\selectfont
    \begin{threeparttable}[b]
        \centering
        % latex table generated in R 4.2.1 by xtable 1.8-4 package
% Sun Jun 23 15:56:55 2024
\begin{tabular}{llllll}
  \toprule
Output & Teaching & Normal Public & Private For Profit & Private Non Profit & Total \\ 
  \midrule
STAC inpatient & 66.98\% & 19.29\% & 10.25\% & 3.48\% & 100\% \\ 
  STAC oupatient & 57.91\% & 10.29\% & 27.13\% & 4.67\% & 100\% \\ 
  Sessions & 50.12\% & 12.7\% & 20.18\% & 16.99\% & 100\% \\ 
  Outpatient Consultations & 77.69\% & 18.64\% & 0.08\% & 3.59\% & 100\% \\ 
  Emergency & 62.02\% & 29.26\% & 6.31\% & 2.41\% & 100\% \\ 
  Follow-up care and Long-term care & 33.5\% & 14.37\% & 27.31\% & 24.82\% & 100\% \\ 
  Home hospitalization & 47.83\% & 10.75\% & 7.46\% & 33.96\% & 100\% \\ 
  Psychiatry stays & 29.65\% & 47.38\% & 9.6\% & 13.37\% & 100\% \\ 
   \bottomrule
\end{tabular}

        \caption{Hospital share of output weighted by the number of hospitals, 2013-2022}
        \begin{tablenotes}[para,flushleft]
            \footnotesize
            For example, the value $a_{ij}$ where $i$ is STAC inpatient and $j$ is teaching hospitals, is calculated by $a_{ij}= \frac{\text{Number of STAC inpatient  in teaching hospitals}}{\text{Share of teaching hospitals}\times \text{Total number of STAC inpatient}}$.
        \end{tablenotes}
    \end{threeparttable}
\end{table}

\subsection{Estimation}

Let $\log(x_{it})$ be the number of nurses in hospital $i$ at time $t$, and
$\log(y_{it})$ denote a vector of output levels. I estimate
\begin{equation}
    \log(x_{it}) = \beta_0 + \beta_1 \log(y_{it}) + \varepsilon_{it}
\end{equation}

First, having performed the regression separately for each type of hospital, it
is without surprise that teaching hospitals have very different coefficients,
as shown in Table \ref{reg_sep}. In addition to the differences in descriptive
statistics from the last section, this intrinsic difference in input demand
functions or equivalently in production function is another sign that teaching
hospitals may not be directly comparable to other types of hospitals. For this
reason, I will exclude teaching hospitals from the subsequent analysis.

\begin{table}\label{tab:reg_sep}
    \centering
    
\begingroup
\centering
\begin{tabular}{lcccc}
   \tabularnewline \midrule \midrule
   Dependent Variable: & \multicolumn{4}{c}{log(ETP\_INF)}\\
                      & Teaching    & Public      & Forprofit   & Nonprofit \\   
   Model:             & (1)         & (2)         & (3)         & (4)\\  
   \midrule
   \emph{Variables}\\
   Constant           & 2.40$^{a}$  & 0.609$^{a}$ & 1.20$^{a}$  & 0.637$^{a}$\\   
                      & (0.338)     & (0.222)     & (0.088)     & (0.142)\\   
   log(SEJHC\_MCO)    & 0.121$^{a}$ & 0.327$^{a}$ & 0.222$^{a}$ & 0.309$^{a}$\\   
                      & (0.045)     & (0.041)     & (0.013)     & (0.030)\\   
   log(SEJHP\_MCO)    & 0.132$^{a}$ & 0.090$^{a}$ & 0.059$^{a}$ & 0.046$^{c}$\\   
                      & (0.031)     & (0.011)     & (0.009)     & (0.025)\\   
   log(SEANCES\_MED)  & 0.081$^{a}$ & 0.062$^{a}$ & 0.080$^{a}$ & 0.121$^{a}$\\   
                      & (0.019)     & (0.007)     & (0.005)     & (0.015)\\   
   log(CONSULT\_EXT)  & 0.019$^{b}$ & 0.024$^{a}$ & -0.008      & 0.006\\   
                      & (0.009)     & (0.006)     & (0.009)     & (0.011)\\   
   log(PASSU)         & 0.053$^{a}$ & 0.003       & 0.034$^{a}$ & 0.031$^{a}$\\   
                      & (0.011)     & (0.008)     & (0.005)     & (0.011)\\   
   log(ENTSSR)        & 0.044$^{a}$ & 0.036$^{a}$ & 0.040$^{a}$ & 0.080$^{a}$\\   
                      & (0.012)     & (0.008)     & (0.006)     & (0.019)\\   
   log(SEJ\_HAD)      & 0.028       & 0.028$^{a}$ & 0.047$^{a}$ & -0.004\\   
                      & (0.029)     & (0.006)     & (0.018)     & (0.021)\\   
   log(SEJ\_PSY)      & 0.034$^{a}$ & 0.085$^{a}$ & 0.082$^{a}$ & 0.073\\   
                      & (0.011)     & (0.012)     & (0.021)     & (0.051)\\   
   \midrule
   \emph{Fit statistics}\\
   Observations       & 1,268       & 5,938       & 5,047       & 3,064\\  
   R$^2$              & 0.828       & 0.873       & 0.723       & 0.707\\  
   \midrule \midrule
   \multicolumn{5}{l}{\emph{Clustered (FI) standard-errors in parentheses}}\\
   \multicolumn{5}{l}{\emph{Signif. Codes: a: 0.01, b: 0.05, c: 0.1}}\\
\end{tabular}
\par\endgroup



    \caption{Separate estimation of input demand function, 2013-2022}
\end{table}

% \textbf{Summarize the input}: explain the reason why we use nurses instead of medical doctors.

% \subsection{Estimation (Pooling)}

% \footnote{On a side note, we filtered the panel such that
%     \begin{enumerate}
%         \item the number of nurses is positive,
%         \item at least one of STAC inpatient, STAC outpatient, Sessions is positive,
%         \item the number of observations is larger than 6
%     \end{enumerate}}

% \begin{table}
%     
\begingroup
\centering
\begin{tabular}{lcccc}
   \tabularnewline \midrule \midrule
   Dependent Variable: & \multicolumn{4}{c}{Nurses}\\
                           & Pool          & Pool IV       & Dummy          & Dummy IV \\   
   Model:                  & (1)           & (2)           & (3)            & (4)\\  
   \midrule
   \emph{Variables}\\
   Constant                & 1.38$^{***}$  & 1.39$^{***}$  & 1.59$^{***}$   & 1.58$^{***}$\\   
                           & (0.022)       & (0.025)       & (0.025)        & (0.027)\\   
   STAC inpatient          & 0.282$^{***}$ & 0.279$^{***}$ & 0.277$^{***}$  & 0.276$^{***}$\\   
                           & (0.005)       & (0.005)       & (0.004)        & (0.005)\\   
   STAC outpatient         & 0.049$^{***}$ & 0.050$^{***}$ & 0.056$^{***}$  & 0.057$^{***}$\\   
                           & (0.003)       & (0.003)       & (0.003)        & (0.004)\\   
   Medical sessions        & 0.062$^{***}$ & 0.061$^{***}$ & 0.064$^{***}$  & 0.063$^{***}$\\   
                           & (0.002)       & (0.002)       & (0.002)        & (0.002)\\   
   External consultations  & 0.052$^{***}$ & 0.057$^{***}$ & 0.024$^{***}$  & 0.027$^{***}$\\   
                           & (0.001)       & (0.002)       & (0.002)        & (0.002)\\   
   Emergency               & 0.019$^{***}$ & 0.016$^{***}$ & 0.022$^{***}$  & 0.021$^{***}$\\   
                           & (0.001)       & (0.001)       & (0.001)        & (0.001)\\   
   Long-term \& follow-up  & 0.076$^{***}$ & 0.076$^{***}$ & 0.068$^{***}$  & 0.069$^{***}$\\   
                           & (0.002)       & (0.002)       & (0.002)        & (0.002)\\   
   Home care               & 0.018$^{***}$ & 0.016$^{***}$ & 0.027$^{***}$  & 0.026$^{***}$\\   
                           & (0.003)       & (0.003)       & (0.003)        & (0.003)\\   
   Psychiatric care        & 0.075$^{***}$ & 0.073$^{***}$ & 0.064$^{***}$  & 0.063$^{***}$\\   
                           & (0.003)       & (0.003)       & (0.003)        & (0.003)\\   
   Private Forprofit       &               &               & -0.282$^{***}$ & -0.270$^{***}$\\   
                           &               &               & (0.023)        & (0.027)\\   
   Private Nonprofit       &               &               & -0.198$^{***}$ & -0.180$^{***}$\\   
                           &               &               & (0.020)        & (0.021)\\   
   STJR0                   &               &               & 0.713$^{***}$  & 0.707$^{***}$\\   
                           &               &               & (0.020)        & (0.021)\\   
   \midrule
   \emph{Fit statistics}\\
   Observations            & 15,335        & 13,402        & 15,335         & 13,402\\  
   R$^2$                   & 0.820         & 0.821         & 0.837          & 0.838\\  
   \midrule \midrule
   \multicolumn{5}{l}{\emph{Heteroskedasticity-robust standard-errors in parentheses}}\\
   \multicolumn{5}{l}{\emph{Signif. Codes: ***: 0.01, **: 0.05, *: 0.1}}\\
\end{tabular}
\par\endgroup



% \end{table}

% \begin{table}
%     
\begingroup
\centering
\begin{tabular}{lcccc}
   \tabularnewline \midrule \midrule
   Dependent Variable: & \multicolumn{4}{c}{log(ETP\_INF)}\\
                      & Teaching    & Public      & Forprofit   & Nonprofit \\   
   Model:             & (1)         & (2)         & (3)         & (4)\\  
   \midrule
   \emph{Variables}\\
   Constant           & 3.28$^{a}$  & 1.38$^{a}$  & 1.40$^{a}$  & 1.00$^{a}$\\   
                      & (0.328)     & (0.262)     & (0.095)     & (0.149)\\   
   log(SEJHC\_MCO)    & 0.108$^{b}$ & 0.331$^{a}$ & 0.261$^{a}$ & 0.344$^{a}$\\   
                      & (0.042)     & (0.048)     & (0.015)     & (0.034)\\   
   log(SEJHP\_MCO)    & 0.132$^{a}$ & 0.078$^{a}$ & 0.048$^{a}$ & 0.046$^{c}$\\   
                      & (0.032)     & (0.013)     & (0.011)     & (0.027)\\   
   log(SEANCES\_MED)  & 0.060$^{a}$ & 0.051$^{a}$ & 0.075$^{a}$ & 0.094$^{a}$\\   
                      & (0.020)     & (0.007)     & (0.006)     & (0.016)\\   
   log(CONSULT\_EXT)  & 0.017       & 0.025$^{a}$ & -0.003      & 0.001\\   
                      & (0.014)     & (0.008)     & (0.011)     & (0.012)\\   
   log(PASSU)         & 0.049$^{a}$ & -0.009      & 0.033$^{a}$ & 0.025$^{b}$\\   
                      & (0.011)     & (0.008)     & (0.005)     & (0.010)\\   
   log(ENTSSR)        & 0.058$^{a}$ & 0.052$^{a}$ & 0.057$^{a}$ & 0.118$^{a}$\\   
                      & (0.013)     & (0.008)     & (0.008)     & (0.019)\\   
   log(SEJ\_HAD)      & 0.022       & 0.028$^{a}$ & 0.049$^{a}$ & -0.011\\   
                      & (0.027)     & (0.007)     & (0.018)     & (0.022)\\   
   log(SEJ\_PSY)      & 0.026$^{b}$ & 0.070$^{a}$ & 0.084$^{a}$ & 0.045\\   
                      & (0.011)     & (0.010)     & (0.018)     & (0.046)\\   
   \midrule
   \emph{Fit statistics}\\
   Observations       & 1,123       & 5,260       & 4,415       & 2,604\\  
   R$^2$              & 0.779       & 0.860       & 0.742       & 0.754\\  
   \midrule \midrule
   \multicolumn{5}{l}{\emph{Clustered (FI) standard-errors in parentheses}}\\
   \multicolumn{5}{l}{\emph{Signif. Codes: a: 0.01, b: 0.05, c: 0.1}}\\
\end{tabular}
\par\endgroup



% \end{table}

% A simple counterfactual to perform:
% % insert a table where we compare the number of nurses needed in public hospitals if they were to use the input demand function of private hospitals.

% \subsection{Estimation (Fixed effect)}
% Though \cite{croiset2024hospitals} has mentioned that identification (and
% significance level) is mainly attributed to \textit{between-group} variation,
% there's still some level of \textit{within-group} variation that guarantee the
% estimation of parameters.

% The strict exogeneity assumption is not always realistic, especially in the
% case of production or factor demand function estimation. Therefore, we resort
% to a series of seminal work in panel data estimation
% \cite{anderson1982formulation,arellano1991some,arellano1995another,blundell1998initial}.

% \begin{table}
%     \label{tab:reg_wg_fd_gmm}
%     
\begingroup
\centering
\begin{tabular}{lccc}
   \tabularnewline \midrule \midrule
   Dependent Variable:                 & \multicolumn{3}{c}{Nurses}                                   \\
                                       & Within Group               & First Difference & System GMM   \\
   Model:                              & (1)                        & (2)              & (3)          \\
   \midrule
   \emph{Variables}                                                                                   \\
   STAC inpatient                      & $0.10^{***}$               & $0.07^{***}$     & $0.51^{***}$ \\
                                       & $(0.00)$                   & $(0.01)$         & $(0.02)$     \\
   STAC outpatient                     & $0.02^{***}$               & $0.01^{***}$     & $0.06^{***}$ \\
                                       & $(0.00)$                   & $(0.00)$         & $(0.02)$     \\
   Medical sessions                    & $0.02^{***}$               & $0.02^{***}$     & $0.04^{***}$ \\
                                       & $(0.00)$                   & $(0.00)$         & $(0.01)$     \\
   External consultations              & $0.00$                     & $0.00$           & $0.07^{***}$ \\
                                       & $(0.00)$                   & $(0.00)$         & $(0.01)$     \\
   Emergency                           & $0.01^{***}$               & $0.01$           & $-0.07^{**}$ \\
                                       & $(0.00)$                   & $(0.00)$         & $(0.03)$     \\
   Long-term              \& follow-up & $0.01^{***}$               & $0.01^{***}$     & $0.02$       \\
                                       & $(0.00)$                   & $(0.00)$         & $(0.02)$     \\
   Home care                           & $0.01^{***}$               & $0.02^{**}$      & $0.01$       \\
                                       & $(0.00)$                   & $(0.01)$         & $(0.02)$     \\
   Psychiatric care                    & $0.02^{***}$               & $0.01$           & $0.04$       \\
                                       & $(0.00)$                   & $(0.01)$         & $(0.03)$     \\
   \midrule

   \midrule
   \emph{Fit statistics}                                                                              \\
   n                                   & $1690$                     & $1690$           & $1690$       \\
   T                                   & $9$                        & $9$              & $9$          \\
   \midrule \midrule
   \multicolumn{4}{l}{\emph{Signif. Codes: ***: 0.01, **: 0.05, *: 0.1}}                              \\
\end{tabular}
\par\endgroup


% \end{table}

% It is worth mentioning that the standard first difference GMM where lagged
% level $x_{i,t-2}$ is used as instruments for the first difference equation
% $\Delta \varepsilon = \Delta y_{i,t}-\Delta x_{i,t} \beta$ does not produce
% results of pure noise. This weak instrument issue was pointed out by
% \cite{blundell1998initial}.

% In the rest of the article, I proceed with the estimation results of system GMM
% as in col 3 of Table \ref{tab:reg_wg_fd_gmm}.

% \section{Compound Decision: The Selection Problem}
% % this is where it finally gets interesting
% % from James Stein to Parametric to Non-parametric. Gives an overview of the history of Empirical Bayes Compound Decision

% \subsection{Compound Decision}
% Leaving the estimation above aside, I will first introduce the idea of compound
% decision pioneered by \cite{robbins1956empirical} before entering into the
% ranking and selection problem.

% Now there are $N$ units, each has an unobserved parameter labelled as
% $(\theta_1,\ldots, \theta_n)$. We are given $N$ estimates
% $(\hat{\theta}_1,\ldots, \hat{\theta}_n) $ of the underlying heterogeneous
% parameters. And each estimate $\hat{\theta}_i$ conditioned on $\theta_i$
% follows a distribution $\hat{\theta}_i | \theta_i \sim P_{\theta_i}$. No matter
% what the specific task is, I care about the collective performance of my
% decision. That being said, I will explicitly define the loss function to
% reflect my attention to the so-called collective performance.

% First the decision rule is defined as a vector of individual decisions
% \begin{equation*}
%     \delta(Y) = (\delta_1(\hat{\theta}), \ldots, \delta_n(\hat{\theta}))
% \end{equation*}
% where each $\delta_i(\cdot)$ is a decision rule for the estimate of $\theta_i$ with the vector $\hat{\theta}$ as input.
% The individual loss for a decision is $L(\theta_i, \delta_i(\hat{\theta}_i))$, giving rise to the aggregated \textbf{compound loss}
% \begin{equation*}
%     L_n(\theta, \delta(\hat{\theta})) = \sum_{i=1}^n L(\theta_i, \delta_i(\hat{\theta})).
% \end{equation*}
% While the compound risk, defined as the expectation of compound loss can be expressed as
% \begin{align*}
%     R_n(\theta, \delta(\hat{\theta})) & = \E[L_n(\theta, \delta(\hat{\theta}))]                                                                                                                                    \\
%                                       & = \frac{1}{n}\sum_{i=1}^n \E[L(\theta_i, \delta_i(\hat{\theta}))]                                                                                                          \\
%                                       & = \frac{1}{n}\sum_{i=1}^n \int \ldots \int L(\theta_i, \delta_i(\hat{\theta}_1, \ldots, \hat{\theta}_n))dP_{\theta_1}(\hat{\theta}_1)\ldots dP_{\theta_n}(\hat{\theta}_n). \\
% \end{align*}

% Given the objective function which is the compound risk, our goal is to find a
% decision rule $\delta(\cdot)$ that minimizes it. This is the optimal compound
% decision rule for a given vector $\hat{\theta}$.
% \begin{equation*}
%     {\delta}^*(\hat{\theta}) =
%     \arg\min_{\delta} R_n(\theta, \delta(\hat{\theta}))
% \end{equation*}

% If $\delta^*(\hat{\theta})$ is separable \footnote{The linear shrinkage class
%     belongs to this class as well. See appendix.}, which means that
% $\delta^*(\hat{\theta})=\{t(\hat{\theta}_1), \ldots, t(\hat{\theta}_n)\}$, the
% compound risk can be written as
% \begin{align*}
%     R_n(\theta, \delta(\hat{\theta})) f & =  \frac{1}{n}\sum \int\ldots\int L(\theta_i, \delta(\hat{\theta}_1,\dots,\hat{\theta}_n))dP_{\theta_1}(\hat{\theta}_1)\ldots dP_{\theta_n}(\hat{\theta}_n) \\
%                                         & = \int_{\theta} \int L(\theta_i, t(\hat{\theta}_i))dP_{\theta_i}(\hat{\theta}_i)dG_n(\theta)                                                                \\
%                                         & =\E_{G_n}{\E_\theta\bra{L(\theta_i, \delta_i(\hat{\theta}))}}                                                                                               \\
% \end{align*}
% where $G_n(\theta)$ is the empirical distribution of $\theta$.\footnote{ $\E_{G_n}\bra{(f(x))} = 1/n \sum_i^n f(x_i)$}

% It is worth mentioning that up til now we treat each $\theta_i$ as fixed
% unknown parameters \footnote{By abuse of terminology, we called this
%     \textit{fixed effect view} while the other assumption \textit{random effect
%         view}. Yet the two terms have nothing to do with whether $\theta_i$ is
%     correlated with $x_{it}$}. However, if we take a Bayesian view on the vector
% $\boldsymbol{\theta}$ by assuming that each $\theta_i$ is an $i.i.d.$ draw from
% an underlying common distribution $G$. The Bayesian risk is
% \begin{equation*}
%     \E_{G}{\E_\theta\bra{L(\theta_i, \delta_i(\hat{\theta}))}}
% \end{equation*}
% The two views are closely linked via the $G_n$ and $G$.

% \begin{quotation}
%     Compound Risk is equivalent to the Bayes risk with prior $G_n$.
% \end{quotation}

% \subsection{Selection Problem}
% Our task at hand is to select the top $\alpha\%$ (e.g. 20\%) of the hospitals
% in terms of labor use efficiency. If $\theta_i$ represents a measure of
% \emph{inefficiency} which is the fixed effect term in the linear input demand
% function specified in Section 2. We want to select the top $\theta_i$s that is
% below than the $\alpha$ quantile of the population $\theta_i<G_n^{-1}(\alpha)$.
% Moreover, we want to subject the selection to another constraint which is the
% False Discovery Rate constraint at level $\gamma$, that is
% \begin{align*}
%     \p\bra{\theta_i>\theta_{\alpha}|\delta_i=1} & =\frac{\p\bra{\theta_i>\theta_{\alpha},\delta_i=1}}{\p\bra{\delta_i=1}}           \\
%                                                 & = \frac{\E_G\bra{1\set{\theta_i>\theta_{\alpha},\delta_i=1}}}{\E_G\bra{\delta_i}} \\
%                                                 & \le \gamma                                                                        \\
% \end{align*}
% Now we are in the position to write down the selection problem subject to the
% capacity constraint at $\alpha$ and FDR constraint at $\gamma$ level, with multiplier $\tau_1$ and $\tau_2$ respectively. We denote
% $\delta_i=1$ when unit $i$ is selected and $h_i=1\{\theta_i<\theta_\alpha\}=1$ when unit $i$ is truly below
% the threshold $\theta_\alpha =G^{-1}(\alpha)$. The compound loss
% function is defined as
% \begin{equation*}
%     L(\delta,\theta)=\sum h_i(1-\delta_i) +\tau_1\pa{\sum (1-h_i)\delta_i -\gamma \delta_i} + \tau_2 \pa{\sum \delta_i -\alpha n}
% \end{equation*}
% To minimize the compound risk is thus
% \begin{align*}
%     \min_{\delta} & \E_G\E_{\theta|\hat{\theta}}\bra{L(\delta,\theta)}                                                                                                        \\
%                   & =\E_G{\sum \E(h_i)(1-\delta_i) +\tau_1\pa{\sum (1-\E(h_i))\delta_i -\gamma \delta_i} + \tau_2 \pa{\sum \delta_i -\alpha n}}                               \\
%                   & =\E_G{\sum v_\alpha(\hat{\theta})(1-\delta_i) +\tau_1\pa{\sum (1-v_\alpha(\hat{\theta}))\delta_i -\gamma \delta_i} + \tau_2 \pa{\sum \delta_i -\alpha n}} \\
% \end{align*}
% where $v_\alpha(\hat{\theta})=\p(\theta<\theta_\alpha|\hat{\theta})$, which we called \textbf{posterior tail probability}. This is in contrast to the posterior mean widely used to shrink the estimate $\hat{\theta}_i$.
% For the moment, it may not immediately obvious how important the prior distribution $G$ is. I will further illustrate it in the next section.
% \subsection{Posterior tail probability}
% For each $\theta_i$, I observe a sequence of $Y_{it}$ coming from a
% longitudinal model
% \begin{equation*}
%     Y_{it} = \theta_i + \varepsilon_{it} \quad \varepsilon_{it} \sim \caln(0,\sigma_i^2) \quad (\theta_i,\sigma_i^2) \sim G
% \end{equation*}
% Neither $\theta_i$ nor $\sigma_i^2$ is known to us. But there are two sufficient statistics for $(\theta_i,\sigma_i^2)$.
% \begin{align*}
%     Y_i=\frac{1}{T_i}\sum_{t=1}^{T_i}Y_{it}           & \quad \text{where}\quad Y_i|\theta_i,\sigma_i^2 \sim \caln(\theta_i,\sigma_i^2/T_i)     \\
%     S_i=\frac{1}{T_i-1}\sum_{t=1}^{T_i}(Y_{it}-Y_i)^2 & \quad \text{where} \quad S_i|\sigma_i^2 \sim \Gamma(r_i= (T_i-1)/2,2\sigma_i^2/(T_i-1)) \\
% \end{align*}

% The tail probability $v_\alpha(y_i,s_i)$ given the two sufficient statistics is
% defined as
% \begin{align*}
%     v_\alpha(Y_i,S_i) & = P( \theta_i > \theta_{\alpha} | Y_i,S_i)                                                                                 \\
%                       & = \frac{{\int_{-\infty}^{\theta_{\alpha}} \Gamma(s_i|r_i,\sigma_i^2) f(y_i|\theta_i, \sigma_i^2) dG(\theta_i,\sigma_i^2)}}
%     {{\int_{-\infty}^{\infty} \Gamma(s_i|r_i,\sigma_i^2) f(y_i|\theta_i, \sigma_i^2) dG(\theta_i,\sigma_i^2)}}
% \end{align*}

% \paragraph{Write out the capacity constraint}
% \paragraph{Write out the FDR constraint}

% If our specification and assumptions on exogeneity are correct, the consistency
% of $\hat{\beta}$ is guaranteed by $N$'s asymptotic. However, our estimate of
% the fixed effect is
% \begin{align*}
%     \hat{\theta}_i & =\frac{1}{T}\sum(\theta_i+\varepsilon_{it}+x_{it}(\beta-\hat{\beta}))              \\
%                    & \overset{N\to \infty}{\longrightarrow} \theta_i+\frac{1}{T}\sum_t \varepsilon_{it} \\
% \end{align*}
% When $T$ is relatively small (or even fixed), I am not in a good position to use central limit theorem to claim that $\hat{\theta}_i \overset{d}{\to} \caln(\theta_i,\frac{\sigma_i^2}{T})$. A bold assumption that $\varepsilon_{it} \sim \caln(0,\sigma_i^2)$ will save me from the $T$ issue, which I will impose for the rest of the section (and abstract from whether that  for each $i$ is a testable/reasonable/feasible assumption).

% \section{Selection results}

% Although \cite{gu2023invidious} has presented the decision rule when
% $(\theta_i,\sigma_i^2)$ are unknown. The application assumes that $\sigma_i^2$
% is known/observable. In this section, I will compare the selection results
% under known variance $\sigma_i^2$ and estimated variance $S_i$.

% \paragraph{Known variance, and 4 rules}
% \paragraph{Unknown variance, and 4 rules}

% \section{Conclusion}

\newpage
\bibliography{ref.bib}

\end{document}
